\subsection{Mathematical Tool}
A set $C$ is a convex set if for any two elements $x_1, x_2 \in C$, $\theta x_1+(1-\theta) x_2\in C$ holds for any $\theta \in \left[0,1\right]$. Loosely speaking, any line segments defined by points which belong to a convex set shall be in the set. A function $f : \mathbf{R^n} \to \mathbf{R}$ is convex if \textbf{dom} $f$ is a convex set and any chord between any two points on $f$ lies above the graph of $f$. A convex optimization (CO) problem has the form 
\begin{align}
\begin{split}
\label{eq:problem}
\text{minimize} \quad  & f_0(x) \\
\text{subject to} \quad & f_i(x) \leq 0,\quad i=1,\ldots,m\\
                  & h_i(x) = 0,\quad i=1,\ldots,p
\end{split}
\end{align}
where all the functions in \ref{eq:problem} are convex functions. $f_0(x)$ is named as objective function, $f_i(x)$ inequality constraint and $h_i(x)$ equality constraint. Although it is now a mature technique to solve a CO problem, it is not easy to formulate objective and constraint functions as convex ones. In the case of mechatronic system design, some typical objective functions, such as volume of the system, are not convex functions and not allowed to be manipulated. However they do follow polynomial form. Luckily, it is still possible to take the advantage of CO technique if those optimization problems can be formulated as another classic form, which is geometric programming (GP).

Before discussion about GP, there are two concepts needed beforehand. A monomial function is defined as \ref{eq:monomial}
\begin{equation}
\label{eq:monomial}
f(x)=cx_1^{a_1}x_2^{a_2}\cdots x_n^{a_n}, \quad \mathbf{dom} \, f=\mathbf{R_{++}^n}
\end{equation}
where $c$ is positive and $a_i$ can be any real number. A posynomial function is a sum of monomials, which is 
\begin{equation}
\label{eq:posynomial}
f(x)=\sum_{k=1}^K c_kx_1^{a_{1k}}x_2^{a_{2k}}\cdots x_n^{a_{nk}}, \quad \mathbf{dom} \, f=\mathbf{R_{++}^n}
\end{equation}
A GP problem follows the same format as \ref{eq:problem}, except that objective and inequality constraints are posynomials and equality constraints are monomials. In fact, by conducting variable substitution, it is easy to show a GP problem can be transferred as a CO problem \cite{boyd2004convex}.

\subsection{Modeling Methodology for Mechanical Components}
GP is considered to be promising in solving optimization problems of mechatronic system design since many of the constraints can be expressed by polynomial approximation. Although there is some distance between posynomials and polynomials, it is possible to get posynomial form while scarificing a little bit accuracy since the parameters we deal with are all in positive sector. Many properties of the mechanical components have been summarized as lookup tables. Posynomial constraints of those elements can be generated by data fitting. In fact, some of the models in \cite{roos2007towards} are achieved in this approach. 
 