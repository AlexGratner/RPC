Roos presented a methodology for the optimal design methodology[???]. According to his proposed methodology, an optimal mechatronic servo system consisting of a motor, planetary gear transmission and load can be achieved regarding system volume, weight and efficiency. The physical system configuration is shown in fig ???. Roos built various dimensioning models for the physical system upon the most critical constraints and designed the corresponding controller by time expensive simulation for minimized control error so that the complete system would be designed at the very early design phase. However, Roos$'$ methodology treats the static model and the dynamic model separately relying on two optimization loops, so the methodology is not very holistic. In fact, only one fixed system configuration: servo actuator was considered and could only be optimized mainly for a single objective in his methodology.

Malmquist et al. extended Roos$'$ work further. A prototype software tool was developed for implementing the methodology. The software allows for arbitrary system configuration including PMDC motor, planetary gear, harmonic drive, solid shaft, linear timing belt drive, load as well as full state feedback controller and multiple objectives can be optimized. With these elementary components, Malmquist presented design examples of haptic steering wheel and two axis gantry system. The methodology is completed to some extent and shows applicable potential to solve more general engineering problems compared with Roos.

The modeling part in Malmquist$'$s work also includes static model and dynamic model. The static model is built for each component based on the critical constraints of the component using objective-relevant design variables and can reflect the relationship between the capability of the component and its design variables. The model is scalable according to the required load. The complexity of the model is aimed at as low as possible so they did not use FEM-analysis for the reason of time efficiency.

The dynamic model in Malmquist$'$s work is used to evaluate the dynamic performance such as maximum integrated square error, maximum error and overshoot etc. Usually dynamics evaluation involves simulation and that would increase complexity as well as time consumption. The method proposed by Malmquist is that the dominant frequencies of the input signal are identified by applying Fourier transforms and
a series of sinusoidal waves would be used to mimic the input signal. Their corresponding outputs would be calculated based on the transfer function separately considering the frequency dependent gain and phase shift. The sum of all the individual output is regarded as the approximation of the time domain response. The transfer function is determined symbolically outside the optimization loop for the reason of time efficiency. During the optimization process of the dynamic part, the controller design without satisfying the dynamic requirements would be rejected. In addition, when the physical system is implemented with a controller, the static model derived earlier might be needed to be checked if it also fulfills the requirements of the dynamic behavior. Thus, a dimensioning factor is introduced to over- or under-dimension the static model so that an optimum could be obtained with both the static and dynamic properties taken into consideration.      

The optimization method employed by Malmquist is genetic algorithm, which is one of the most widely used evolutionary methods. It is a non-gradient based optimization. The algorithm is inspired by the Darwinian principle of natural selection. The possible solution with design variables can be compared to an individual with a number of genes. The optimum would be reached after breeding several generations. This method can handle a diverse range of problems but at a cost of low computation efficiency. So replacing it with a more efficient optimization method such as convex optimization is worth being studied further.