
The current trend in research related to early design optimization of mechatronic systems is to use genetic algorithms to minimize the objective functions, but these approaches suffers from the potential disadvantages of genetic algorithms which could be avoided by using the convex optimization approach discussed in this paper. The multidisciplinary nature of mechatronic systems, which could be characterised as systems with synergistic integrations of mechanical engineering with electronics and intelligent computer control \cite{MechatronicsWWHHarashima}, makes early design optimization a difficult process. Optimizations are often done in the latter detailed design phases \cite{EngineeringDesign} whereas early design decisions becomes a limiting factor. Research on the topic of optimization in early design stages for mechatronic systems has presented both holistic and non-holistic approaches which share the common feature of using genetic algorithms, either as a complement or as a standalone solver. The disadvantages of GA when it comes to accuracy and computation time justifies research on applying disciplinary convex optimization.


\par
Optimizing mechatronic systems using genetic algorithms has been widely explored by researchers today. Hammadi et al. \cite{Hammadi2012, Hammadi2014} propose an emergent multi-agent approach where the system is decomposed into smal sub-systems (or agents) that in turn is optimized using the genetic algorithms NSGA II. Another approach that uses NSGA II is proposed by Guizani \cite{Guizani2014} where a partitioning method is utilized to decompose the mechatronic system and classification of the interactions between partions are done. An approach developed by Seo \cite{Seo2003} and further developed by Behbahani \cite{Behbahani2013, Behbahani2014} uses a two loop optimization process which incorporates genetic algorithms and bond graphs in an outer loop to optimize system topology and genetic programming in an inner loop to fin the elite solution within the optimal topology. The research in this paper will be based upon the work done by Malmquiest et al. \cite{malmquist2015tool}, they introduce a tool called IDIOM for holistic optimization of mechatronic design concept by utilizing genetic algorithms.

\par
The IDIOM framework developed by Malmquist et al. extends on the Roos \cite{roos2007} on a research on a methodology for integrated design and mechatronic servo systems. In his research, Roos reflects on the choice of using genetic algorithms to optimize the servo systems by stating that "The drawback of this method for system optimization is that it is computationally intensive and that there exists no mathematical proof that it actually finds the true optimum". This statement justifies the exploration of other optimization techniques which could provide true optimizations in a less computationally intensive manner, and for which convex optimization has been chosen. Disciplinary convex programming is a methodology developed by Grant et al. \cite{gb08} which purpose, according to the authors, is to "allow much of the manipulation and transformation required to analyse and solve convex programs to be automated". The methodology originates from procedures taken by those who regularly study convex optimization problems and has been implemented in the modelling framework called \textit{CVX} \cite{cvx} which serves as a package for specifying and solving convex programs.

\par
The argument made by Roos serve as a starting point in the research presented in this report. It challenges the current trend of using evolutionary algorithms when optimizing the design of mechatronic systems and could provide a optimization approach that is less computationally expensive and more accurate. 

\par
The remainder of this paper will introduce the previous work done by Malmquist et al. of which this study aim to extend on. Paragraph 3 describes the method used when applying convex optimization to the mechatronic system. Finally, paragraph 4 introduces a case study where a system optimization is performed, and compared to the results obtained by Malmquist et al., of a mechatronic system composed of a motor, planetary gear and a load.  


